% !TEX spellcheck = de_DE,en_US
\documentclass{book}

\usepackage{csquotes}

\begin{document}
  Project Gutenberg's Also Sprach Zarathustra, by Friedrich Wilhelm Nietzsche

https://www.gutenberg.org/ebooks/7205n
  Friedrich Nietzsche
\title{Also sprach Zarathustra}

  Ein Buch für Alle und Keinen



  Erster Theil

  \chapter{Zarathustra's Vorrede}

  Als Zarathustra dreissig Jahr alt war, verliess er seine Heimat und den See seiner Heimat und ging in das Gebirge. Hier genoss er seines Geistes und
  seiner Einsamkeit und wurde dessen zehn Jahr nicht müde. Endlich aber verwandelte sich sein Herz, - und eines Morgens stand er mit der Morgenröthe
  auf, trat vor die Sonne hin und sprach zu ihr also:

  \begin{displayquote}[Also begann Zarathustra's Untergang]
    Du grosses Gestirn! Was wäre dein Glück, wenn du nicht Die hättest, welchen du leuchtest!

    Zehn Jahre kamst du hier herauf zu meiner Höhle: du würdest deines Lichtes und dieses Weges satt geworden sein, ohne mich, meinen Adler und meine
    Schlange.

    Aber wir warteten deiner an jedem Morgen, nahmen dir deinen Überfluss ab und segneten dich dafür.

    Siehe! Ich bin meiner Weisheit überdrüssig, wie die Biene, die des Honigs zu viel gesammelt hat, ich bedarf der Hände, die sich ausstrecken.

    Ich möchte verschenken und austheilen, bis die Weisen unter den Menschen wieder einmal ihrer Thorheit und die Armen einmal ihres Reichthums froh
    geworden sind.

    Dazu muss ich in die Tiefe steigen: wie du des Abends thust, wenn du hinter das Meer gehst und noch der Unterwelt Licht bringst, du überreiches
    Gestirn!

    Ich muss, gleich dir, untergehen, wie die Menschen es nennen, zu denen ich hinab will.

    So segne mich denn, du ruhiges Auge, das ohne Neid auch ein allzugrosses Glück sehen kann!

    Segne den Becher, welcher überfliessen will, dass das Wasser golden aus ihm fliesse und überallhin den Abglanz deiner Wonne trage!

    Siehe! Dieser Becher will wieder leer werden, und Zarathustra will wieder Mensch werden.
  \end{displayquote}
\end{document}
